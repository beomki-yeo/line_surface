\documentclass[12pt,a4paper]{scrarticle}

\usepackage{theme/acts}

\usepackage{lipsum}

\title{A whitepaper about a topic}
\author{First Author \and Second Author}


\begin{document}

\maketitle

\begin{abstract}
    %% Only use basic LaTeX markup here, it gets rendered by MathJax.

Line surface used for wire measurements or perigee surface is an unique type of surface where the intersections of a track is not defined on a physical plane. Due to this trait, the underlying mathematics gets quite complicated especially with coordinate transform jacobian. Unfortunately, ACTS does not have a right reference on this, and the author also could not locate proper one (Please let the authors know if anyone finds it).In this white paper, thus we attempt to derive the jacobians for both local-to-global and global-to-local transforms.
\end{abstract}

% Covariance and Jacobian
\newcommand{\jacobianLocalToGlobal}{\mathbb{J}_{G}}
\newcommand{\jacobianGlobalToLocal}{\mathbb{J}_{L}}

% Global parameter
\newcommand{\globalPosition}{\mathbf{r}}
\newcommand{\globalDirection}{\mathbf{t}}
\newcommand{\qoverp}{\psi}

% Local
\newcommand{\localPosition}{\mathbf{l}}
\newcommand{\localAngle}{\boldsymbol{\lambda}}
\newcommand{\surfaceCenter}{\mathbf{c}}
\newcommand{\surfaceNormal}{\mathbf{w}}
\newcommand{\ubasis}{\mathbf{u}}
\newcommand{\vbasis}{\mathbf{v}}
\newcommand{\localPosU}{l_0}
\newcommand{\localPosV}{l_1}

% Etc
\newcommand{\zeroMatrix}{\mathbb{O}}

\tableofcontents

\section{Coordinate definition}
 The global coordinate consists of 3 dimension cartesian position ($\globalPosition$), 3 dimension unit cartesian direction ($\globalDirection$), and a charge divided by momentum ($\qoverp$) \footnote{A time component will be skipped in this white paper}, hence it has 7 parameters in total. Meanwhile the local coordinate is described by two local positions ($\localPosition$), two angle parameters ($\localAngle$) and $\qoverp$. The local position is made of the closest distance ($l_0$) between the track and line and longitudinal position ($l_1$) along the line axis. The sign of $l_0$ is defined as the sign of $ (\globalDirection \times \surfaceNormal) \cdot (\surfaceCenter - \globalPosition)$, where $\surfaceNormal$ is a unit vector along the line axis and $\surfaceCenter$ is the center of the line. The two angle parameters are $\phi$ and $\theta$ in the global spherical coordinate.
 
\section{Local to global transformation}

A jacobian for local to global transformation is represented as the derivative of global coordinate with respect to the local coordinate:

\begin{equation}\label{eq:jacobi}
    \jacobianLocalToGlobal = 
    \begin{pmatrix}
     \frac{\partial{\globalPosition}}{\partial{\localPosition}} & \frac{\partial{\globalPosition}}{\partial{\localAngle}} & \frac{\partial{\globalPosition}}{\partial{\qoverp}} \\
     \frac{\partial{\globalDirection}}{\partial{\localPosition}} & \frac{\partial{\globalDirection}}{\partial{\localAngle}} & \frac{\partial{\globalDirection}}{\partial{\qoverp}} \\
    \frac{\partial{\qoverp}}{\partial{\localPosition}} & \frac{\partial{\qoverp}}{\partial{\localAngle}} & \frac{\partial{\qoverp}}{\partial{\qoverp}}
    \end{pmatrix}.
\end{equation}

Let's take a heuristic approach first to identify the zero sub-matrices. $\frac{\partial{\globalPosition}}{\partial{\qoverp}}$ and $\frac{\partial{\globalDirection}}{\partial{\qoverp}}$ are zero because the intersection points do not change with the variation of the momentum. $\frac{\partial{\qoverp}}{\partial{\localPosition}}$ and $\frac{\partial{\qoverp}}{\partial{\localAngle}}$ are also zero for the same reason. $\frac{\partial{\globalDirection}}{\partial{\localPosition}}$ also should be zero because track direction should not change when $\lambda$ is fixed. Summing up, the jacobian can be simplified as the following:

\begin{equation}\label{eq:jacobi2}
    \jacobianLocalToGlobal = 
    \begin{pmatrix}
     \frac{\partial{\globalPosition}}{\partial{\localPosition}} & \frac{\partial{\globalPosition}}{\partial{\localAngle}} &  \zeroMatrix \\
     \zeroMatrix & \frac{\partial{\globalDirection}}{\partial{\localAngle}} &  \zeroMatrix \\
     \zeroMatrix & \zeroMatrix & 1
    \end{pmatrix},
\end{equation}
where $\zeroMatrix$ was used as a notation for zero sub-matrices.

It is straightforward to obtain $\frac{\partial{\globalDirection}}{\partial{\localAngle}}$ because $\globalDirection$ can be represented as trigonometric functions of $\phi$ and $\theta$. $\frac{\partial{\globalPosition}}{\partial{\localPosition}}$ can be calculated by defining a cartesian reference frame defined with $\ubasis$, $\vbasis$, and $\surfaceNormal$ unit vectors:
\begin{align}\label{eq:new_coord}
    \ubasis & = \frac{\vbasis \times \globalDirection}{|\vbasis \times \globalDirection|}, \nonumber \\
    \surfaceNormal & =  \ubasis \times \vbasis.
\end{align}

Since $l_0$ and $l_1$ is defined on $\ubasis$ and $\vbasis$, 
we can relate $\globalPosition$ with $\localPosition$ as following: 
\begin{equation}\label{eq:r_in_uv}
    \globalPosition = l_0 \ubasis + l_1 \vbasis +  \surfaceCenter.
\end{equation}

It is straightforward to show that
$\frac{\partial{\globalPosition}}{\partial{\localPosition}}$ is the following $3 \times 2$ matrix: 
\begin{equation}
\frac{\partial{\globalPosition}}{\partial{\localPosition}}=
    \begin{pmatrix}
     \ubasis & \vbasis 
    \end{pmatrix}    
\end{equation}

$\frac{\partial{\globalPosition}}{\partial{\localAngle}}$ can be calculated by making use of the following equations:
\begin{equation}\label{eq:eq1}
    \frac{\partial \globalDirection}{\partial \phi} \cdot \globalDirection = 0,
\end{equation}

\begin{equation}\label{eq:eq2}
    \vbasis \times \globalDirection \times \vbasis = \globalDirection - \vbasis(\globalDirection \cdot \vbasis),
\end{equation}

\begin{equation}\label{eq:eq3}
    \surfaceNormal \cdot \globalDirection = \frac{1- (\vbasis \cdot \globalDirection)^2}{|\vbasis \times \globalDirection|} = |\vbasis \times \globalDirection|,
\end{equation}

\begin{align}\label{eq:eq4}
   \frac{\partial |\vbasis \times \globalDirection|}{\partial \phi} & = \frac{\partial \sin{\xi}}{\partial \phi} \nonumber \\
   & = \cos{\xi} \frac{\partial \xi}{\partial \phi} \nonumber \\
   & = \vbasis \cdot \globalDirection \frac{\partial \cos^{-1}{\vbasis \cdot \globalDirection}}{\partial \phi} \nonumber \\
   & = - \frac{\vbasis \cdot \globalDirection}{|\vbasis \times \globalDirection|} \vbasis \cdot \frac{\partial \globalDirection}{\partial \phi},
\end{align}
where $\xi$ is the angle between $\vbasis$ and $\globalDirection$. Note that eq. (\ref{eq:eq1})-(\ref{eq:eq4}) also hold for $\theta$. Then let us derive $\frac{\partial{\globalPosition } }{\partial{\localAngle }}$ using eq. (\ref{eq:new_coord})-(\ref{eq:eq4}):

\begin{eqnarray}\label{eq:drdphi}
    \frac{\partial{\globalPosition}}{\partial \phi} & = &  l_0 \frac{\partial}{\partial \phi} \frac{\vbasis \times \globalDirection}{|\vbasis \times \globalDirection|} \nonumber \\
    & = & l_0 \left[ - \frac{\vbasis \times \globalDirection}{|\vbasis \times \globalDirection|^2} \frac{\partial |\vbasis \times \globalDirection|}{\partial \phi} + \frac{1}{|\vbasis \times \globalDirection|} \vbasis \times \frac{\partial \globalDirection}{\partial \phi} \right] \nonumber \\
    & = & \frac{l_0}{|\vbasis \times \globalDirection|}\left[ -\ubasis \left( -\frac{\vbasis \cdot \globalDirection}{|\vbasis \times \globalDirection|} \vbasis \cdot \frac{\partial \globalDirection}{\partial \phi} \right)  + \vbasis \times \frac{\partial \globalDirection}{\partial \phi}  \right] \nonumber \\
    & = & \frac{l_0}{|\vbasis \times \globalDirection|}\left[ -\ubasis \frac{1}{|\vbasis \times \globalDirection|} \left( \globalDirection - (\vbasis \cdot \globalDirection)\vbasis \right) \cdot \frac{\partial \globalDirection}{\partial \phi} + \vbasis \times \frac{\partial \globalDirection}{\partial \phi}  \right] \nonumber \\
    & = & \frac{l_0}{|\vbasis \times \globalDirection|}\left[ -\ubasis \left( \vbasis \cdot \frac{\partial \globalDirection}{\partial \phi} \right)  + \vbasis \times \frac{\partial \globalDirection}{\partial \phi}  \right] \nonumber \\
    & = & \frac{l_0}{|\vbasis \times \globalDirection|}\left[ -\ubasis \left( \ubasis \times \vbasis \cdot \frac{\partial \globalDirection}{\partial \phi} \right)  + \vbasis \times \frac{\partial \globalDirection}{\partial \phi}  \right] \nonumber \\   
    & = & \frac{l_0}{\surfaceNormal \cdot \globalDirection}\left[ -\ubasis \left( \ubasis \cdot \vbasis \times \frac{\partial \globalDirection}{\partial \phi} \right)  + \vbasis \times \frac{\partial \globalDirection}{\partial \phi}  \right].    
\end{eqnarray}
$\frac{\partial \globalPosition}{\partial \theta}$ follows the same equation.

\section{Global to local transformation}

A Jacobian for global to local transformation is the following:

\begin{equation}\label{eq:jacobi3}
    \jacobianGlobalToLocal = 
    \begin{pmatrix}
     \frac{\partial{\localPosition}}{\partial{\globalPosition}} & 
     \frac{\partial{\localPosition}}{\partial{\globalDirection}}&  \frac{\partial{\localPosition}}{\partial{\qoverp}} \\
     \frac{\partial{\localAngle}}{\partial{\globalPosition}} & 
     \frac{\partial{\localAngle}}{\partial{\globalDirection}}&  \frac{\partial{\localAngle}}{\partial{\qoverp}} \\
     \frac{\partial{\qoverp}}{\partial{\globalPosition}} & 
     \frac{\partial{\qoverp}}{\partial{\globalDirection}}&  \frac{\partial{\qoverp}}{\partial{\qoverp}}
    \end{pmatrix},
\end{equation}

As for the local to global transformation, the off-diagonal partial derivatives related to $\qoverp$ are zero matrices:

\begin{equation}\label{eq:jacobi4}
    \jacobianGlobalToLocal = 
    \begin{pmatrix}
     \frac{\partial{\localPosition}}{\partial{\globalPosition}} & 
     \frac{\partial{\localPosition}}{\partial{\globalDirection}}&  \zeroMatrix \\
     \frac{\partial{\localAngle}}{\partial{\globalPosition}} & 
     \frac{\partial{\localAngle}}{\partial{\globalDirection}}&  \zeroMatrix \\
     \zeroMatrix & 
     \zeroMatrix &  1
    \end{pmatrix},
\end{equation}

$\frac{\partial{\localAngle}}{\partial{\globalPosition}}$ should also be zero matrix because $\globalDirection$ is fixed. $\frac{\partial{\localPosition}}{\partial{\globalDirection}}$ should also be zero matrix because $\globalPosition$ is fixed. 

\begin{equation}\label{eq:jacobi5}
    \jacobianGlobalToLocal = 
    \begin{pmatrix}
     \frac{\partial{\localPosition}}{\partial{\globalPosition}} & \zeroMatrix &  \zeroMatrix \\
     \zeroMatrix & \frac{\partial{\localAngle}}{\partial{\globalDirection}} &  \zeroMatrix \\
     \zeroMatrix & \zeroMatrix & 1
    \end{pmatrix},
\end{equation}

$\frac{\partial{\localAngle}}{\partial{\globalDirection}}$ is just a inverse of $\frac{\partial{\globalDirection}}{\partial{\localAngle}}$, and $\frac{\partial{\localPosition}}{\partial{\globalPosition}}$ is the transpose of $\frac{\partial{\globalPosition}}{\partial{\localPosition}}$.

\begin{equation}\label{eq:dldr}
\frac{\partial{\localPosition}}{\partial{\globalPosition}}=
    \begin{pmatrix}
     \ubasis^{\mathrm{T}} \\
     \vbasis^{\mathrm{T}}
    \end{pmatrix}.    
\end{equation}

If it is not clear, it is helpful to take an inner product between Eq. (\ref{eq:r_in_uv}) and $\ubasis$ or $\vbasis$, and to perform the differentiation.

\section{Discussions}

$\frac{\partial{\globalPosition}}{\partial{\localAngle}}$ is the only nonzero submatrix, which makes it worth discussing how the partial derivative works here. When $l_0$ and $l_1$ is fixed, there is only one way that $\globalPosition$ can vary, which is a rotation around $\vbasis$. In other words, $d\globalPosition$ is parallel to $\surfaceNormal$. In the case of $\frac{\partial{\localPosition}}{\partial{\globalDirection}}$, $\globalDirection$ just rotates around $\ubasis$, which does not introduce any change in $\localPosition$.

It is easy to show that $\jacobianGlobalToLocal\jacobianLocalToGlobal$ is an identity matrix. What we need to prove is that $\frac{\partial{\localPosition}}{\partial{\globalPosition}} \frac{\partial{\globalPosition}}{\partial{\localAngle}}$ is zero matrix.  Our discussion above already implies that $\frac{\partial{\globalPosition}}{\partial{\localAngle}}$ is orthogonal to $\ubasis$ and $\vbasis$, and the orthogonality can be confirmed from Eq. (\ref{eq:drdphi}). Therefore, $\frac{\partial{\localPosition}}{\partial{\globalPosition}} \frac{\partial{\globalPosition}}{\partial{\localAngle}}$ is a zero matrix because $\frac{\partial{\localPosition}}{\partial{\globalPosition}}$ consists of $\ubasis$ and $\vbasis$ according to Eq. (\ref{eq:dldr}).

\end{document}
